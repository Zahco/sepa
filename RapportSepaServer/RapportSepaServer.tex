\documentclass{article}

\usepackage[utf8]{inputenc}
\usepackage[T1]{fontenc}
\usepackage{hyperref}
\usepackage{tabularx}
\usepackage{array}
\usepackage{fancyhdr}
\usepackage{graphicx}
\usepackage[a4paper]{geometry}
\usepackage{multicol}
\usepackage{listings}

\title{Rapport du projet SEPA \\ sepa-server}
\author{par Geoffrey SPAUR et Camille LEPLUMEY}
\date{26 avril 2017}
\pagestyle{fancy}
\lhead{Rapport du projet SEPA - sepa-server \\ \textbf{M1GIL} - Geoffrey SPAUR et Camille LEPLUMEY}
\rhead{\includegraphics[scale=0.5]{logo_univ_rouen.png}}
\setlength{\headsep}{1cm}
\begin{document}

\maketitle
\newpage
\tableofcontents{}
\newpage
\section{Information minimals}
\subsection{Auteurs}
Le projet a été réalisé par Camille LEPLUMEY et Geoffrey SPAUR. Ce projet a pour but de mettre en place une API REST 
permettant la gestion de transactions bancaires.
\subsection{Adresse du service REST}
Vous trouverez l’adresse de notre API ici: \url{https://gscl-sepa.herokuapp.com/}.
\newpage

\section{Description du serveur}
\subsection{Adresse du service REST}
Vous trouverez l’adresse de notre API ici: \url{https://gscl-sepa.herokuapp.com/}.
\subsection{Description précises des requêtes}
\subsection{Liste des technologies}
Durant ce projet, nous avons utilisé toutes les technologies vues en cours. Cependant il a été nécessaire 
d'utiliser des technologies avant-gardiste.
Notamment pour effectuer la persistance des données et le déploiement de notre service sur une plateforme Cloud.
\subsubsection{Déploiement du service}
Comme conseillé par notre chargé de TP, nous avons utilisé la plateforme \href{https://www.heroku.com}{Heroku}.
Par conséquent nous avons utilisé le CLI afin d'uploader notre projet et de le déployer automatiquement.
Dans un premier vous devez vous créer et activer un compte sur \href{https://www.heroku.com}{Heroku}. Puis vous pouvez créer une application:
\begin{lstlisting}[language=bash]
heroku create <app_name>
\end{lstlisting}
Nous avons choisi pour nom d'application: gscl-sepa.

\subsubsection{Persistance des données}
\subsection{Tutoriel de déploiement}
\newpage

\section{ Description du client}
\subsection{Tutoriel d'installation et d’exécution}
\subsection{Mode d’emploi}
\subsection{Exemple de fichiers}
\end{document}