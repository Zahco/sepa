\documentclass{article}

\usepackage[utf8]{inputenc}
\usepackage[T1]{fontenc}
\usepackage{hyperref}
\usepackage{tabularx}
\usepackage{array}
\usepackage{fancyhdr}
\usepackage{graphicx}
\usepackage[a4paper]{geometry}
\usepackage{multicol}
\usepackage{listings}

\title{Rapport du projet SEPA \\ sepa-server}
\author{par Geoffrey SPAUR et Camille LEPLUMEY}
\date{26 avril 2017}
\pagestyle{fancy}
\lhead{Rapport du projet SEPA - sepa-server \\ \textbf{M1GIL} - Geoffrey SPAUR et Camille LEPLUMEY}
\rhead{\includegraphics[scale=0.5]{logo_univ_rouen.png}}
\setlength{\headsep}{1cm}
\begin{document}

  \maketitle
  \newpage
  \tableofcontents{}
  \newpage
  
  \section{Information minimales}
    \subsection{Auteurs}
      Le projet a été réalisé par Camille LEPLUMEY et Geoffrey SPAUR. Ce projet a pour but de mettre en place une API REST 
      permettant la gestion de transactions bancaires.
    \subsection{Adresse du service REST}
      Vous trouverez l’adresse de notre API ici: \url{https://gscl-sepa.herokuapp.com/}.
  \newpage

  \section{Description du serveur}
    \subsection{Adresse du service REST}
      Vous trouverez l’adresse de notre API ici: \url{https://gscl-sepa.herokuapp.com/}.
    \subsection{Description précise des requêtes}
      \subsubsection{Resume}
	\emph{Resume} permet d'obtenir une courte description des transactions stockées en base.\\
	\textbf{GET} \url{https://gscl-sepa.herokuapp.com/resume}
	\paragraph{Retour}:
	  \begin{lstlisting}[language=xml]
      <transactionResumes>
	<transactionResume>
	  <date>1992-03-05</date>
	  <ident>Example</ident>
	  <montant>12322.2999999999992724</montant>
	  <num>SL0001</num>
	</transactionResume>
	<transactionResume>
	  <date>1992-03-05</date>
	  <ident>Example</ident>
	  <montant>12322.2999999999992724</montant>
	  <num>SL0002</num>
	</transactionResume>
      </transactionResumes>
	    \end{lstlisting}
      \subsubsection{Home}
	\emph{Home} permet d'obtenir des informations sur le projet tels que le nom des auteurs.\\
	\textbf{GET} \url{https://gscl-sepa.herokuapp.com/accueil}
	\paragraph{Retour}:
	  \begin{lstlisting}[language=xml]
      <welcomeXML>
	<auteurs>Geoffrey SPAUR, Camille LEPLUMEY</auteurs>
	<date>30/07/04/2017</date>
      </welcomeXML>
	  \end{lstlisting}
	
      \subsubsection{Reset}
	\emph{Reset Database} permet de supprimer et de recréer la base de données.Cette fonction est utilisée afin de supprimer les données corrompues. \\
	\textbf{GET} \url{https://gscl-sepa.herokuapp.com/reset}
	\paragraph{Retour}:
	  \begin{lstlisting}[language=xml]
      <message>Database restored</message>
	  \end{lstlisting}
	
	\subsubsection{Test}
	\emph{Test} permet d'ajouter un exemple de transaction dans la base de données. \\
	\textbf{GET} \url{https://gscl-sepa.herokuapp.com/test}
	\paragraph{Retour}:
	  \begin{lstlisting}[language=xml]
      <ns2:CstmrDrctDbtInitn>
	<ns2:DrctDbtTxInf>
	  <PmtId>Example</PmtId>
	  <InstdAmt Ccy="EUR">12322.2999999999992724</InstdAmt>
	  <ns2:DrctDbtTx>
	    <MndtId>FF-57-E7</MndtId>
	    <DtOfSgntr>1992-03-05</DtOfSgntr>
	  </ns2:DrctDbtTx>
	  <ns2:DbtrAgt>
	    <BIC>DAEOFRPPCCT</BIC>
	  </ns2:DbtrAgt>
	  <ns2:Dbtr>
	    <Nm>Mr Dupont</Nm>
	  </ns2:Dbtr>
	  <ns2:DbtrAcct>
	    <IBAN>FE1405643596877A</IBAN>
	  </ns2:DbtrAcct>
	  <RmtInf>Ceci est un commentaire</RmtInf>
	</ns2:DrctDbtTxInf>
      </ns2:CstmrDrctDbtInitn>
	  \end{lstlisting}
	
      \subsubsection{Stats}
	\emph{Stats} permet d'obtenir les statistiques concernant la base de données.\\
	\textbf{GET} \url{https://gscl-sepa.herokuapp.com/stats}
	\paragraph{Retour}:
	  \begin{lstlisting}[language=xml]
      <transactionStats>
	<nbOfTransaction>0</nbOfTransaction>
	<totalAmount>0</totalAmount>
      </transactionStats>
	    \end{lstlisting}
      \subsubsection{Depot}
	\emph{Depot} permet d'ajouter une transaction.\\
	\textbf{POST} \url{https://gscl-sepa.herokuapp.com/depot}
	\paragraph{Paramètres}:
	  \begin{lstlisting}[language=xml]
      <ns2:CstmrDrctDbtInitn>
	<ns2:DrctDbtTxInf>
	  <PmtId>Example</PmtId>
	  <InstdAmt Ccy="EUR">12322.2999999999992724</InstdAmt>
	  <ns2:DrctDbtTx>
	    <MndtId>FF-57-E7</MndtId>
	    <DtOfSgntr>1992-03-05</DtOfSgntr>
	  </ns2:DrctDbtTx>
	  <ns2:DbtrAgt>
	    <BIC>DAEOFRPPCCT</BIC>
	  </ns2:DbtrAgt>
	  <ns2:Dbtr>
	    <Nm>Mr Dupont</Nm>
	  </ns2:Dbtr>
	  <ns2:DbtrAcct>
	    <IBAN>FE1405643596877A</IBAN>
	  </ns2:DbtrAcct>
	  <RmtInf>Ceci est un commentaire</RmtInf>
	</ns2:DrctDbtTxInf>
      </ns2:CstmrDrctDbtInitn>
	  \end{lstlisting}
	\paragraph{Retour}:
	  \begin{lstlisting}[language=xml]
      <ns2:CstmrDrctDbtInitn>
	<ns2:DrctDbtTxInf>
	  <PmtId>Example</PmtId>
	  <InstdAmt Ccy="EUR">12322.2999999999992724</InstdAmt>
	  <ns2:DrctDbtTx>
	    <MndtId>FF-57-E7</MndtId>
	    <DtOfSgntr>1992-03-05</DtOfSgntr>
	  </ns2:DrctDbtTx>
	  <ns2:DbtrAgt>
	    <BIC>DAEOFRPPCCT</BIC>
	  </ns2:DbtrAgt>
	  <ns2:Dbtr>
	    <Nm>Mr Dupont</Nm>
	  </ns2:Dbtr>
	  <ns2:DbtrAcct>
	    <IBAN>FE1405643596877A</IBAN>
	  </ns2:DbtrAcct>
	  <RmtInf>Ceci est un commentaire</RmtInf>
	</ns2:DrctDbtTxInf>
      </ns2:CstmrDrctDbtInitn>
	  \end{lstlisting}
      \subsubsection{Transaction}
	\emph{Transaction} permet de consulter en détails une transaction.\\
	\textbf{GET} \url{https://gscl-sepa.herokuapp.com/trx/{id}}
	\paragraph{Paramètres}:
	  \begin{lstlisting}[language=xml]
      id : int
	  \end{lstlisting}
	\paragraph{Retour}:
	  \begin{lstlisting}[language=xml]
      <ns2:CstmrDrctDbtInitn>
	<ns2:DrctDbtTxInf>
	  <PmtId>Example</PmtId>
	  <InstdAmt Ccy="EUR">12322.2999999999992724</InstdAmt>
	  <ns2:DrctDbtTx>
	    <MndtId>FF-57-E7</MndtId>
	    <DtOfSgntr>1992-03-05</DtOfSgntr>
	  </ns2:DrctDbtTx>
	  <ns2:DbtrAgt>
	    <BIC>DAEOFRPPCCT</BIC>
	  </ns2:DbtrAgt>
	  <ns2:Dbtr>
	    <Nm>Mr Dupont</Nm>
	  </ns2:Dbtr>
	  <ns2:DbtrAcct>
	    <IBAN>FE1405643596877A</IBAN>
	  </ns2:DbtrAcct>
	  <RmtInf>Ceci est un commentaire</RmtInf>
	</ns2:DrctDbtTxInf>
      </ns2:CstmrDrctDbtInitn>
	  \end{lstlisting}
	  
      \subsubsection{Delete}
	\emph{Delete} permet de supprimer une transaction donnée. \\
	\textbf{DELETE} \url{https://gscl-sepa.herokuapp.com/delete/{id}}
	\paragraph{Paramètres}:
	  \begin{lstlisting}[language=xml]
      id : int
	  \end{lstlisting}
	\paragraph{Retour}:
	  \begin{lstlisting}[language=xml]
      <message>Transaction {id} deleted</message>
	  \end{lstlisting}
    
    \subsection{Liste des technologies}
      Durant ce projet, nous avons utilisé toutes les technologies vues en cours. Cependant il a été nécessaire 
      d'utiliser des technologies avant-gardiste.
      Notamment pour effectuer la persistance des données et le déploiement de notre service sur une plateforme Cloud.
      \subsubsection{Déploiement du service}
	Comme conseillé par notre chargé de TP, nous avons utilisé la plateforme \href{https://www.heroku.com}{Heroku}.
	Par conséquent nous avons utilisé le CLI afin d'\emph{uploader} notre projet et de le déployer automatiquement.
	Dans un premier vous devez vous créer et activer un compte sur \href{https://www.heroku.com}{Heroku}. Puis vous pouvez créer une application:
	\begin{lstlisting}[language=bash]
    $ heroku create <app_name>
	\end{lstlisting}
	Nous avons choisi pour nom d'application: gscl-sepa.
	Notre projet est une application \emph{maven} générant un fichier \emph{war}. Ce fichier peut être déployé sur  \emph{tomcat} ou  \emph{glassfish}. Afin de générer
	le fichier  \emph{war}, il suffira d'entrer la commande suivante à la racine de notre projet:
	\begin{lstlisting}[language=bash]
    $ mvn install
	\end{lstlisting}
	Le fichier  \emph{war} se trouvera dans le dossier \emph{target}. \\
	Dans un premier temps, nous avons vérifié que notre application ne produisait pas d'erreurs en la testant en local sur  \emph{glassfish}.
	Puis nous l'avons déployée sur Heroku dans un  \emph{tomcat} comme conseillé par les \href{https://devcenter.heroku.com/articles/war-deployment}{tutoriels} proposés par Heroku.
	Pour cela nous avons installé le plugin proposé par Heroku:
	\begin{lstlisting}[language=bash]
    $ heroku plugins:install heroku-cli-deploy
	\end{lstlisting}
	Puis nous pouvons déployer notre application:
	\begin{lstlisting}[language=bash]
    $ heroku war:deploy <path_to_war_file> --app <app_name>
	\end{lstlisting}
	Pour finir vous pourrez accéder à l'application avec cette url:\\ \emph{\url{https://<app_name>.herokuapp.com}}.

      \subsubsection{Persistance des données}
	\paragraph{Instanciation de la base}
	  Avant de pouvoir sauvegarder nos transactions, nous avons besoin de créer notre base de données.
	  Nous avons pour cela utilisé \href{https://elements.heroku.com/addons/heroku-postgresql}{une application postgresql proposée par Heroku}.
	  En créant cette application nous avons eut accès à une database pré-créée.
	  Afin d'effectuer nos tests, nous avons installé un plugin permettant l'accès à notre base de données:
	  \begin{lstlisting}[language=bash]
    $ heroku addons:create heroku-postgresql:hobby-dev
	  \end{lstlisting}
	  Enfin nous nous connectons à la base de données avec la commande suivante:
	  \begin{lstlisting}[language=bash]
    $ heroku pg:psql --app <app_name>
	  \end{lstlisting}
	  Nous avons effectué divers tests: création de tables, insertion de données...\\
	  Après nos tests, nous avons créé notre base de données:
	  \begin{lstlisting}[language=sql]
    create table transaction (id serial primary key, data xml);
	  \end{lstlisting}
	  Le type \emph{serial} combine un type entier avec l'option \emph{auto-increment}. Le type \emph{XML} permet de sauvegarder des données sous format XML. Ce dernier type permettra 
	  d'utiliser des fonctions XML dans nos requêtes SQL, telles que la fonction \emph{xpath()}.
	\paragraph{Lien avec l'application}
	  Maintenant que notre base est instanciée, nous pouvons établir une connexion entre cette dernière et notre application.
	  Pour obtenir toutes les informations de connexion, vous pouvez entrer la commande suivante:
	  \begin{lstlisting}[language=sql]
    heroku pg:credentials --app <app_name>
	  \end{lstlisting}
	  Maintenant nous allons configurer notre application pour la connecter à notre base de données. Pour cela il nous faut utiliser un \emph{driver} spécifique pour \emph{postgresql}. 
	  Nous utilisons la dépendance suivante (à ajouter dans le fichier \emph{pom.xml}).
	  \begin{lstlisting}[language=xml]
    <dependency>
      <groupId>org.postgresql</groupId>
      <artifactId>postgresql</artifactId>
      <version>9.4-1201-jdbc4</version>
    </dependency>
	  \end{lstlisting}
	  Ensuite pour établir la connexion, il faudra charger le driver puis utiliser nos \emph{credentials}.
	  \begin{lstlisting}[language=java]
    Class.forName("org.postgresql.Driver");
    Connection connection =  
	DriverManager.getConnection(url, user, password);
	  \end{lstlisting}
	  Il est à noter que l'url doit être de cette forme:
	  \begin{lstlisting}[language=bash]
    jdbc:postgresql://<host>:<port>/<database>
	  \end{lstlisting}
	  Vous retrouverez ces informations dans le fichier \emph{/src/main/java/rest/DAO/ClassDAO.java}.
    \subsection{Tutoriel de déploiement}
      Dans cette section, nous allons voir - de manière spécifique - comment déployer notre serveur REST.
      Dans un premier temps, vous devez vous rendre sur notre dépôt GitHub afin de télécharger les sources du projet:\\
      \url{https://github.com/Zahco/sepa}\\
      Puis vous devrez générer notre fichier \emph{war}:
      \begin{lstlisting}[language=bash]
    $ mvn install
      \end{lstlisting}
      Libre à vous d'utiliser le service (\emph{tomcat}, \emph{glassfish}) qui vous sied. Pour notre part nous utilisons la plateforme \href{https://www.heroku.com}{Heroku}.
      Vous devrez donc \href{https://devcenter.heroku.com/articles/heroku-cli}{télécharger et installer le cli}, afin de pouvoir vous connecter à la plateforme en notre nom.
      \begin{lstlisting}[language=bash]
    $ heroku login
    Enter your Heroku credentials:
    Email: geoffrey.spaur@gmail.com
    Password: aqwzsxedc.2017
      \end{lstlisting}
      Une fois connecté vous pourrez déployer notre application comme ceci:
      \begin{lstlisting}[language=bash]
    $ heroku war:deploy target/sepa-server.war --app gscl-sepa
      \end{lstlisting}
      Enfin ouvrez notre application dans votre navigateur ou utilisez notre client.
      \begin{lstlisting}[language=bash]
    $ heroku open
      \end{lstlisting}
      Vous trouverez l’adresse de notre API ici: \url{https://gscl-sepa.herokuapp.com/}.

  \newpage

  \section{ Description du client}
    \subsection{Tutoriel d'installation et d’exécution}
      Afin d'utiliser notre client vous devrez le télécharger ici:\\
      \url{https://github.com/Zahco/sepa-client}\\
      Compiler les sources: 
      \begin{lstlisting}[language=bash]
    $ mvn install
      \end{lstlisting}
      Exécuter le fichier jar:
      \begin{lstlisting}[language=bash]
    $ java -jar target/sepa-client.jar
      \end{lstlisting}
    \subsection{Mode d’emploi}
      \paragraph{}
	Vous trouverez dans la partie haute du client différents boutons parlant d'eux même. 
	\begin{itemize}
	 \item \emph{Home}: Pour accéder à divers informations
	 \item \emph{Stat}: Pour afficher les statistiques.
	 \item \emph{Resume}: Pour obtenir un résumé de chaque transactions.
	 \item \emph{Clear}: Pour vider la console.
	 \item \emph{Reset Database}: Pour vider la base de données.
	\end{itemize}

      \paragraph{}
	Vous trouverez dans la partie gauche du client un formulaire permettant d'ajouter une transaction personnalisée.
	Attention, cependant les champs sont soumis à la norme ISO 20022, notamment les champs BIC et IBAN.
	Vous pourrez envoyer le formulaire en cliquant sur le bouton: \emph{Depot}.
      \paragraph{}
	Dans le centre de l'application, vous trouverez la console permettant d'observer les réponses du serveur.
      \paragraph{}
	Enfin dans la partie basse du client, vous pouvez rechercher une transaction pour son identifiant.
	Grâce à cette recherche, vous obtiendrez toutes les informations concernant la transaction.
	Pour obtenir un identifiant, il vous suffira d'obtenir la liste des transactions avec l'option \emph{Resume}.
	Rechercher l'attribut \emph{transactionResume.num} en retirant les 2 premières lettres.
    \subsection{Exemple de fichiers}
      Vous pouvez ajouter une transaction "exemple" en accédant à l'url suivante: \\
      \url{https://gscl-sepa.herokuapp.com/test}
      
  \newpage
  \section{Fonctions complémentaires}
    \paragraph{Clear}
      Afin de faciliter la lecture en levant toute ambiguïté, nous avons rajouté un bouton \emph{Clear} dans
      notre client afin de vider la console.
    \paragraph{Reset Database}
      Durant le développement du projet, nous avons utilisé un Git public. Pensant que personne ne s'intéresserait à nous,
      nous avons librement disposé les accès à la base de données. Seulement certain ont décidé de jouer avec. 
      En ajoutant directement des lignes en base de données, cela corrompt le modèle XML. Nous avons ajouté le bouton
      \emph{Rest Database} pour nettoyer la base de données. Nous avons par-ailleurs caché les mots de passe...
    \paragraph{Delete}
      La fonction delete permet de consommer une transaction dans la base de données (cf: sujet du projet).
      L'identifiant correspond à la colonne id en bise de données. Vous retrouverez cet identifiant avec la commande \emph{resume},
      en recherchant l'attribut \emph{transactionResume.num} et en retirant les 2 premières lettres.
\end{document}